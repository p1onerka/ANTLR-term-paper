\documentclass{article}
\usepackage[russian]{babel}
\usepackage[letterpaper,top=2cm,bottom=2cm,left=3cm,right=3cm,marginparwidth=1.75cm]{geometry}
\usepackage{amsmath}
\usepackage{graphicx}
\usepackage[colorlinks=true, allcolors=blue]{hyperref}

\title{Расширение функциональности генератора синтаксических анализаторов ANTLR на язык программирования OCaml}
\author{Котельникова Ксения Андреевна}

\begin{document}
\maketitle

\section{Вступление}

В настоящее время существует множество задач, для которых необходимы специфические синтаксические анализаторы. К таким задачам относятся разработка компиляторов и интерпретаторов, обработка пользовательских конфигурационных файлов, создание предметно-ориентированных языков программирования, статический анализ кода и прочие. При этом, самостоятельная разработка анализатора требует от программиста (жаргон?) знаний специальной теории, а зачастую и специальных языков программирования. Кроме того, даже грамотный разработчик потратит на такой проект внушительное (какое?) количество времени. Чтобы решить эту проблему, были внедрены (изобретены?) инструменты, называемые генераторами синтаксических анализаторов. Их задача состоит в формировании кода анализатора на основе запроса пользователя, оформленного в специальном синтаксисе, а также в предоставлении пользователю легкого в освоении интерфейса. На данный момент такие инструменты нашли широкое применение в индустрии и вытеснили аналогичную ручную разработку (голословно?)

Одним из самых распространенных генераторов является ANTLR (Another Tool for Language Recognition). На данный момент он официально поддерживает восемь императивных (ооп? а с++ номинально и функциональный) языков программирования, а также имеет множество пользовательских дополнений для интеграции других. 

Благодаря вариативности и легкости применения, при разработке синтаксического анализатора использование ANTLR часто является требованием компании. При этом, актуальная его версия не поддерживает ни одного полноценно функционального языка. Это может помешать дальнейшей интеграции кода в проект, так как в задачах, для которых может понадобиться предметно-ориентированный синтаксический анализатор, часто является целесообразным использование функциональных языков программирования (объяснить, почему?)

Таким образом была поставлена задача о расширении функциональности ANTLR на какой-либо из общепринятых (?) функциональных языков. В качестве целевого языка был выбран OCaml благодаря его сходству с другими языками, что позволяет адаптировать код с минимальными усилиями.

\section{Обзор}
\begin{enumerate}
    \item ANTLR (сравнение с Bison, Cup, Lark, основной аргумент -- много выходных языков)
    \item OCaml (назвать всякие F#, Haskell, ReasonML похожими на него)
    \item Dune (используется внутри либы, сравнить с OCamlBuild, make, jbuilder (?))
\end{enumerate}
\end{document}