% !TeX spellcheck = ru_RU
% !TEX root = vkr.tex

\section{Обзор}
\label{sec:relatedworks}

Обзор проводился с целью установления преимущества выбранного инструмента по сравнению с аналогами, а также для изучения технологий, необходимых для реализации расширения ANTLR.

\subsection{Обзор используемых технологий}

\subsubsection{OCaml}

Генерация кода на OCaml удобна в первую очередь, потому что позволяет бесшовно соединить полученный анализатор с основным телом проекта. 
Язык был выбран из соображения его распространенности в сфере синтаксического анализа.
Такую популярность OCaml получил благодаря возможности использования как инструментов объектно-ориентированного программирования, так и преимуществ функциональной парадигмы. 
Среди этих преимуществ, например, мощная система типов, позволяющая быстро определять некорректные объявления в коде, а также возможность реализовывать функции высших порядков, которые делают код более декларативным, коротким и легким для понимания.
Со временем, к этим удобствам прибавилась накопившаяся база библиотек, разработанных специально для синтаксического анализа и значительно упрощающих разработку. 

\subsubsection{ANTLR}

\begin{itemize}
    \item Архитектура
    
    ANTLR можно разделить на 3 логические части. 
    Первая включает в себя обработку исходного файла, в котором находится описание грамматики (для четвертой версии ANTLR это файлы с типом .g4). 
    Результатом его является дерево абстрактного синтаксиса лексического и синтаксического анализа. 
    В своей работе я не затрагиваю эту часть, она является общей для всех выходных языков.
    
    Вторая часть включает в себя генерацию файлов Lexer, Parser и нескольких вспомогательных файлов на основе деревьев абстрактного синтаксиса и специального шаблона из библиотеки StringTemplate. 
    При помощи API, предоставляемого библиотекой, в шаблон подставляются значения из дерева абстрактного синтаксиса, что на выходе формирует полноценный файл на целевом языке.
    
    Однако в таком виде синтаксический анализатор еще не готов к работе. 
    Для обработки обработки входной строки ему необходима третья часть инструмента: вспомогательная функциональность, которая заранее должна быть реализована внутри ANTLR в модуле runtime. 
    Особенность модуля заключается в том, что он должен быть реализован полностью на целевом языке, и содержать все необходимые для компиляции конфигурационные файлы. 
    Фактически, runtime должен представлять собой полноценную библиотеку вспомогательных для синтаксического анализатора методов.  
    \item Библиотека StringTemplate

    Генерация файлов синтаксического и лексического анализаторов происходит посредством библиотеки StringTemplate. 
    С помощью нее можно создавать так называемые строковые шаблоны ~--- текст со специально обозначенными местами, в которые нужно будет вставить некоторое поступившее значение. 
    Аргументы, которые будут подставляться в шаблон, называются аттрибутами. 
    
    Библиотека StringTemplate предоставляет исчерпывающий API для взаимодействия с шаблонами, но в контексте ANTLR важнее всего метод add, позволяющий по шаблону, от которого он вызывается, сгенерировать готовую строку с аргументами метода, подставленными вместо аттрибутов. 

    Еще одной важной функциональностью в StringTemplate является создание групп шаблонов. 
    Это файлы с типом .stg, внутри которых можно объявить несколько взаимосвязанных шаблонов. 
    Внутри одной группы шаблоны могут подставляться друг в друга, передавая свои значения аттрибутов в качестве аргументов внутреннему шаблону. 
    Именно так и происходит создание синтаксического анализатора: генерация основного класса запускает множество вложенных генераций. 
    Таким образом, для добавления нового целевого языка необходимо описать группу шаблонов, на основе которых будет создаваться результат.
    \item Сравнение с другими аналогичными инструментами

    \begin{itemize}
        \item В отличие от YACC и Bison, генерирующих LALR (lookahead left-to-right) анализаторы, ANTLR поддерживает LL(*) (leftmost derivation с разбором слева направо и заглядыванием вперед на произвольное количество токенов). 
        Для достижения произвольности количества токенов ANTLR использует конечные автоматы вместо таблиц переходов. 
        В отличие от LR подхода, LL не поддерживает обработку грамматик с леворекурсивными правилами. 
        Чтобы решить эту проблему, ANTLR сканирует введенную в него грамматику на наличие левой рекурсии, и, при обнаружении, переписывает эти правила в эквивалентные с использованием префиксов. 
        Такая система неидеальна: слишком сложные правила ANTLR не способен преобразовать автоматически. 
        В этом случае он уведомит пользователя о необходимости переписать правило. 
        LL-анализ, однако, имеет серьезное преимущество перед LR в обработке ошибок и удобстве использования. 
        \item ANTLR ~--- самый популярный инструмент (количество звезд на гитхабе не аргументирует популярность?) для генерации синтаксических анализаторов, и за время его существования накопилось много материалов, помогающих разобраться в его работе. 
        Это выгодно выделяет его на фоне YACC, Bison, PEG.js и других менее популярных решений.
    \end{itemize}
\end{itemize}

\subsubsection{Dune}

Модуль runtime должен быть реализован на целевом языке, что, в случае OCaml, ставит вопрос о выборе системы сборки. 
Dune из всех возможных вариантов является самой легкой в использовании, наряду с OCamlBuild, Jbuilder и OASIS. 
Кроме того, на данный момент, Dune является стандартом среди OCaml-проектов.

\subsection{Обзор существующих решений}

\subsubsection{Реализованные языки ANTLR}

Текущая версия ANTLR%
\footnote{Версия 4.13.2 на момент обращения \DTMdate{2024-12-04}}. поддерживает генерацию парсеров в 10 языков программирования: C++, C\#, Dart, Java, JavaScript, PHP, Python версии 3, Swift, TypeScript и Go. 
Несмотря на наличие черт функциональной парадигмы в некоторых из них, все эти языки далеки по синтаксису и особенностям реализации от OCaml, что делает затруднительным простой перевод готового анализатора с одного из уже реализованных языков на выбранный целевой. 
Кроме того, для работы сгенерированный при помощи ANTLR анализатор требует специально реализованный на выходном языке модуль runtime. 

Эти обстоятельства делают нецелесообразной идею использования ANTLR для OCaml-проекта без расширения текущей функциональности инструмента.

\subsection{Выводы}

На данный момент ANTLR является самым широко используемым генератором синтаксических анализаторов, в том числе, по причине удобства расширения. 
Библиотека StringTemplate помогает полностью освободить разработчика нового выходного языка от необходимости реализовывать все модули, связанные с первичной обработкой файла грамматики. 
Работа, таким образом, сводится к правильной настройке строкового шаблона и реализации модуля runtime, необходимого для работы готового анализатора.