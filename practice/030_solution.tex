% !TeX spellcheck = ru_RU
% !TEX root = vkr.tex

\section{Описание решения}

\subsection{Подготовка файла группы шаблонов}
\label{subsec:string_template_group}

Первая существенная часть работы заключалась в реализации файла группы шаблонов. 
В инструкции по добавлению новых целевых языков от авторов ANTLR присутствует рекомендация о том, 
что за основу нового файла следует взять один из уже реализованных. 
В случае OCaml это не настолько выгодно, так как синтаксис слишком отличается от всех представленных в ANTLR языков, 
но все еще полезно для отслеживания зависимостей шаблонов и общего понимания их смысла. 

Работа с группой шаблонов серьезно замедлялась из-за отсутствия инструментов форматирования и статического анализа файлов типа .stg. 
Ориентироваться в коде было затруднительно, а возникающие ошибки выявлялись только после тестового запуска, на который уходило около 
полуминуты на сборку проекта (или конкретное время — это лишнее?).

Кроме того, сообщения об ошибках не всегда были информативны. 
В библиотеке StringTemplate есть встроенная система обработки ошибок, однако иногда ее сообщения давали неправильное представление 
о возникшей ошибке. 
Например, ошибка, описанная как отсутствие определенного шаблона, на самом деле произошла из-за лишнего управляющего символа 
в совершенно другом шаблоне. 

Также в ANTLR при взаимодействии с файлом группы шаблонов никак не обрабатываются исключения null-указателя 
(или можно написать null pointer exception и не изобретать термины?). 
Это затрудняло локализацию ошибки, так как без дополнительных инструментов невозможно было понять, какой именно объект 
не смог создаться и вызвал исключение. 

В процессе работы пришлось изучить документацию StringTemplate из-за необходимости использовать внутри шаблона символ «<», 
являющийся в этой библиотеке управляющим. 

Готовый файл группы шаблонов был помещен в директорию tool/templates/codegen/OCaml, как и было описано в инструкции 
по добавлению новых целевых языков.

\subsection{Реализация модуля Runtime}
\label{subsec:runtime_realization}

\subsection{Мелкие задачи}
\label{subsec:misc}
\subsubsection{Добавление нового файла target}

