% !TeX spellcheck = ru_RU
% !TEX root = vkr.tex

\section{Описание решения}

\subsection{Подготовка файла группы шаблонов}
\label{subsec:string_template_group}

Первая существенная часть работы заключается в реализации файла группы шаблонов. 
В инструкции по добавлению новых целевых языков от авторов ANTLR присутствует рекомендация о том, что за основу нового файла следует взять один из уже реализованных. 
В случае OCaml это не настолько выгодно, так как синтаксис слишком отличается от всех представленных в ANTLR языков, но все еще полезно для отслеживания зависимостей шаблонов и общего понимания их смысла. 

Работа с группой шаблонов серьезно замедляется из-за отсутствия инструментов форматирования и статического анализа файлов типа .stg. 
Ориентироваться в коде затруднительно, а возникающие ошибки выявляются только после тестового запуска, на который уходит около полуминуты на сборку проекта.

Кроме того, сообщения об ошибках, которые выдает ANTLR, не всегда информативны. 
В библиотеке StringTemplate есть встроенная система обработки ошибок, однако иногда ее сообщения давали неправильное представление о возникшей ошибке. 
Например, ошибка, описанная как отсутствие определенного шаблона, на самом деле произошла из-за лишнего управляющего символа в совершенно другом шаблоне. 

Также в ANTLR при взаимодействии с файлом группы шаблонов никак не обрабатывается исключение разыменования нулевого указателя. 
Это затрудняет локализацию ошибки, так как без дополнительных инструментов невозможно понять, какой именно объект не смог создаться и вызвал исключение. 

В процессе работы пришлось изучить документацию StringTemplate из-за необходимости использовать внутри шаблона символ «<», являющийся в этой библиотеке управляющим. 

Готовый файл группы шаблонов был помещен в директорию tool/templates/codegen/OCaml, как и было описано в инструкции по добавлению новых целевых языков.

\subsection{Реализация модуля runtime}
\label{subsec:runtime_realization}

В ANTLR модуль runtime представляет собой обширную библиотеку, включающую в себя следующие большие логические блоки: работа с ATN (augmented transition networks) и детерминированными конечными автоматами, построение на их основе дерева абстрактного синтаксиса, работа с потоком токенов и классы Lexer и Parser, в которых реализована функциональность, общая для всех сгенерированных на основе грамматики модулей. 

Основная проблема на этом этапе работы состояла в сложности погружения в код. 
Runtime практически не имеет файловой структуры, а та, которая есть, не всегда логически интуитивна. 
Кроме того, на код в большинстве отсутствует документация. 
Ни названия классов, ни какие-либо другие признаки не помогают разбраться в зависимости частей кода друг от друга, а отсутствие какого-либо разделения на блоки приводит к высокой сложности определения взаимосвязи между классами. 
Говоря о названиях, они зачастую еще больше уменьшали читаемость. 

В ходе решения задачи была реализована часть модуля runtime, необходимая для поддержки работы простых числовых грамматик. 
В дальнейшем возможно расширение библиотеки до полной функциональности, представленной в частях ANTLR для других языков.

\subsection{Добавление нового файла target}
\label{subsec:misc}

Для того, чтобы сконфигурировать ANTLR для работы с новым целевым языком, необходимо создать файл с классом OCamlTarget, в котором указать ключевые слова языка и управляющие последовательности. 
На первую часть имени этого класса ANTLR будет ориентироваться при поиске файла строкового шаблона, поэтому шаблон необходимо было назвать соответственно: OCaml.stg.

