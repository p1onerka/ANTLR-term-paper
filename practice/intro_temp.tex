\documentclass{article}
\usepackage[russian]{babel}
\usepackage[letterpaper,top=2cm,bottom=2cm,left=3cm,right=3cm,marginparwidth=1.75cm]{geometry}
\usepackage{amsmath}
\usepackage{graphicx}
\usepackage[colorlinks=true, allcolors=blue]{hyperref}

\title{Расширение функциональности генератора синтаксических анализаторов ANTLR на язык программирования OCaml}
\author{Котельникова Ксения Андреевна}

\begin{document}
\maketitle

\section{Введение}

В настоящее время существует множество задач, для которых необходимы специфические синтаксические анализаторы. 
К таким задачам относятся разработка компиляторов и интерпретаторов, обработка пользовательских конфигурационных файлов, создание предметно-ориентированных языков программирования, статический анализ кода и прочие. 
При этом, самостоятельная разработка синтаксического анализатора требует внушительного (какого?) количества времени.
При таком подходе также увеличивается вероятность пропуска ошибок и недочетов в проекте, и на их предотвращение и исправление следует закладывать еще больше времени.
Чтобы решить эту проблему, были созданы инструменты, называемые генераторами синтаксических анализаторов. 
Их задача состоит в формировании кода анализатора на основе запроса пользователя, оформленного в специальном синтаксисе, а также в предоставлении пользователю легкого в освоении интерфейса. 
Все тестирование при этом происходит внутри инструмента, абстрагированно от пользователя.
Он после прохождения проверок получает сразу пригодный к использованию синтаксический анализатор.
Стоит еще отметить, что инструменты для создания анализаторов часто поддерживают при генерации хорошие практики, про которые при ручном написании легко забыть, такие, например, как выведение понятных сообщений об ошибках анализа.

Одним из самых распространенных генераторов является ANTLR (Another Tool for Language Recognition). 
На данный момент он официально поддерживает восемь императивных языков программирования, а также имеет множество пользовательских дополнений для интеграции других. 

Благодаря легкости применения и обилию поддерживаемых выходных (?) языков, при разработке синтаксического анализатора использование ANTLR часто является требованием компании. 
При этом, актуальная его версия не поддерживает ни одного полноценно функционального языка. 
Это делает инструмент менее гибким, а также не позволяет использовать в работе полезные особенности функциональных языков, которые могли бы повысить производительность итогового анализатора.
Кроме того, интеграция императивного кода анализатора в функциональный код остального проекта может оказаться сложным и трудозатратным процессом (или убрать эту мотивацию вообще?).

Таким образом была поставлена задача о расширении функциональности ANTLR на какой-либо из функциональных языков. 
В качестве целевого языка был выбран OCaml благодаря наличию в нем одновременно и элементов объектно-ориентированной парадигмы, которые упростят его интеграцию в код ANTLR, 
и возможностей функциональной парадигмы, которые позволят сделать результат более эффективным.

\section{Обзор}
\begin{enumerate}
    \item ANTLR (сравнение с Bison, Cup, Lark, основной аргумент -- много выходных языков)
    \item OCaml (назвать всякие F#, Haskell, ReasonML похожими на него)
    \item Dune (используется внутри либы, сравнить с OCamlBuild, make, jbuilder (?))
\end{enumerate}
\end{document}